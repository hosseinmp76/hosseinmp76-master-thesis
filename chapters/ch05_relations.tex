\chapter{
	ارتباط با مسائل مختلف ریاضی
}
متن کلی

\section{
	حدس کلاه
}

\section{
	سرور توزیع شده
}

\section{
	بازآرایی داده
}

در این پژوهش، محققان بر این باورند که استفاده از کدگذاری شاخص انعطاف‌پذیر می‌تواند چارچوبی کارآمدتر و مفیدتر برای تبادل داده‌ها در سیستم‌های محاسبات توزیع‌شده فراهم کند. این رویکرد می‌تواند از انتخاب‌های متعدد تبادل برای کاهش تعداد انتقالات بهره ببرد و به بهبود کارایی ارتباط در این سیستم‌ها کمک کند. از جمله کاربردهای مهم تبادل داده، یادگیری ماشینی توزیع‌شده بر روی داده‌های حجیم است که نیاز به تبادل داده‌های محلی برای آموزش مدل‌های مقاوم‌تر دارد.

پژوهش حاضر تمرکز خود را بر مدل محاسبات توزیع‌شده "استاد-کارگران" قرار داده است که در آن یک گره استاد به \(n\) گره کارگر متصل است و هر کارگر دارای حافظه‌ای برای ذخیره‌سازی داده‌ها است. در هر تکرار، کارگران بر اساس داده‌های موجود در حافظه‌شان محاسبات محلی انجام می‌دهند و استاد با جمع‌آوری و ترکیب نتایج محلی، نتیجه کلی را به دست می‌آورد. سپس، استاد با ارسال پیام‌های جدید، داده‌های موجود در حافظه هر کارگر را به‌روزرسانی می‌کند.

یکی از مشاهدات مهم در این پژوهش این است که گزینه‌های متعددی برای تبادل داده‌ها وجود دارد که می‌توانند عملکرد یکسانی را ارائه دهند. به همین دلیل، پژوهشگران به دنبال طراحی یک ترکیب‌بند نیمه‌تصادفی به همراه طرح کدگذاری هستند تا هزینه‌های ارتباطی را به طور قابل توجهی کاهش دهند. برای ارزیابی کیفیت تبادل داده‌ها، معیار میانگین فاصله همینگ معرفی شده است که تفاوت محتوای ذخیره‌شده در کارگران و بین تکرارها را نشان می‌دهد.

نوآوری اصلی این پژوهش طراحی یک طرح تبادل داده‌ها و کدگذاری نیمه‌تصادفی برای محاسبات توزیع‌شده است که یک سطح مطلوب از میانگین فاصله همینگ را تضمین می‌کند و بر اساس دو تغییر عمده در طراحی کدگذاری شاخص انعطاف‌پذیر بنا شده است. این پژوهش نشان می‌دهد که طرح پیشنهادی می‌تواند در مقایسه با طرح‌های تبادل داده‌های مبتنی بر کدگذاری شاخص، تا 87\% در انتقالات صرفه‌جویی کند، با این حال 2\% افت عملکرد را در نرخ خطا تجربه می‌کند.


\subsection*{مدل سیستم محاسبات توزیع‌شده:}
در مقاله مورد بررسی، سیستم محاسبات توزیع‌شده "استاد-کارگران" با یک گره استاد دارای \(m\) پیام و \(n\) گره کارگر مورد بحث قرار می‌گیرد. هر کارگر با یک کش برای ذخیره‌سازی پیام‌ها مجهز است و گره استاد می‌تواند ارسال‌های پخش بدون خطا را به تمام کارگران انجام دهد. سیستم هدف از حل یک وظیفه محاسباتی را دنبال می‌کند و این کار از طریق تکرارها صورت می‌پذیرد که در هر تکرار، کارگران محاسبات محلی را انجام داده و نتایج محلی را به استاد باز می‌گردانند. سپس استاد با ترکیب نتایج محلی، نتیجه کلی را به دست آورده و برای تکرار بعدی به کارگران پخش می‌کند.

\subsection*{معیار عملکرد برای تبادل داده‌ها:}
برای سنجش اثر تبادل داده‌های نیمه‌تصادفی، معیار میانگین فاصله همینگ بر اساس تفاوت پیام‌های ذخیره‌شده در کارگران و بین تکرارها استفاده می‌شود. فاصله همینگ بین دو نشانگر، تعداد موقعیت‌هایی است که ورودی‌ها برای آن دو نشانگر متفاوت هستند. میانگین فاصله همینگ برای یک طرح تبادل به عنوان میانگین فاصله همینگ در طول زمان و بین گره‌های کارگر تعریف می‌شود.

\begin{theorem}{ (کدگذاری شاخص انعطاف‌پذیر محدود):}
	کدگذاری شاخص انعطاف‌پذیر محدود، که یک جزء اصلی طرح تبادل داده‌های سلسله مراتبی است، به عنوان یک مسئله NP-hard معرفی شده است. در این مسئله، هدف کمینه کردن تعداد انتقالات پخش برای رضایت تمام مشتریان تحت محدودیت \(c\) است: هر پیام باید توسط حداکثر \(c\) مشتریانی که این پیام را درخواست کرده‌اند، رمزگشایی و کش شود. نتایج ارائه شده نشان می‌دهد که تعداد انتقالات پخش برای یک نمونه تصادفی با محدودیت \(c\) تقریبا به طور قطع توسط \(O(\min\{\frac{c}{n} \log(n), \frac{n}{\log(m)}\})\) برای \(c = o(n^{1/7} \log^2(n))\) و \(O(\min\{\frac{n}{c} + \log(c), \frac{n}{\log(m)}\})\) برای \(c = \Omega(n^{1/7} \log^2(n))\) محدود می‌شود.
\end{theorem}

\begin{theorem}{ (طرح تبادل داده‌های سلسله مراتبی):}
	برای تضمین فاصله همینگ کافی به طور متوسط در بین تکرارها و کارگران، یک طرح معماری دو لایه برای تبادل داده‌ها پیشنهاد شده است: لایه بیرونی پیام‌ها را به گروه‌ها تقسیم می‌کند و محتوای کش هر کارگر را به پیام‌های در گروه‌های خاص محدود می‌کند؛ لایه داخلی کدگذاری شاخص انعطاف‌پذیر محدود را برای هر گروه پیام و کارگران مرتبط اعمال می‌کند. این ساختار سلسله مراتبی امکان دریافت پیام‌های جدید از گروه‌های مختلف پیام را برای هر کارگر فراهم می‌کند و به این ترتیب، کارگر می‌تواند کش خود را با یک فراکسیون از پیام‌های جدید بین تکرارها تازه کند. این طرح برای مواردی که \(m\) پیام و \(n\) کارگر با حافظه یکسان وجود دارند توضیح داده شده است.
\end{theorem}
