% !TeX spellcheck = fa-IR
\chapter{اثبات لم تاثیر نداشتن الفبا روی ناحیه ظرفیت}
\label{appendix:l1}

\begin{proof}
	فرض کنید
	$I$
	و
	$I'$
	دو نمونه از 
	\icod
	باشند که روی الفباهای متناهی
	$\chi$
	و
	$\chi'$
	تعریف شده اند و
	$\mathscr{A}$
	و
	$ \mathscr{A}'$
	مجموعه‌ی نرخ‌های قابل دستیابی متناظر با آن‌ها باشند.
	دو حالت در نظر می‌گیریم:
	
	حالت اول: اگر
	$\log_{\card{\chi}} \card{\chi'} = \dfrac{a}{b}: a, b \in \IN$
	یک عدد گویا باشد، برای این‌که نشان دهیم که ناحیه ظرفیت دو نمونه یکی است کافی است نشان دهیم که
	$\mathscr{A} = \mathscr{A}'$.
 فرض کنید
	$R = (R_1, \ldots, R_n) \in \mathscr{A}$.
 در نتیجه طبق تعریف یک 
	$(t, r) = (t_1, \ldots, t_n, r)$-کد
	برای مسئله‌ی
	$I$
	وجود دارد که
	$\forall i \in [n]: R_i \leq t_i / r$
	حال اگر کد بالا را
	$a$
	مرتبه تکرار کنیم تا یک
	$(at, ar)$-کد
	به دست بیاوریم چون هر دو نمونه روی یک مجموعه اطلاعات جانبی تعریف شده‌اند و 
	$\card{\chi}^a = \card{\chi'}^b$
	در نتیجه به یک
	$(bt, br)$
	برای نمونه
	$I'$
	رسیده‌ایم. پس
	$R \in \mathscr{A}'$
	و در نتیجه
	$\mathscr{A} \subseteq \mathscr{A}'$
	با عوض کردن جای دو نمونه و تکرار بحث بالا به
		$\mathscr{A}' \subseteq \mathscr{A}$
		می‌رسیم و اثبات کامل می‌شود.
		
		حالت دوم: اگر
		$\log_{\card{\chi}} \card{\chi'} $
		عددی گنگ باشد. ابتدا نشان می‌دهیم که
		$\mathscr{A}' \subseteq \mathscr{C}_{\chi}$.
  فرض کنید که
		$R \in \mathscr{A}'$
		پس طبق تعریف یک
		$(t, r)$-کد
		وجود دارد برای
		$I'$
		وجود دارد که
		$\forall i \in [n]: R_i \leq \dfrac{t_i}{r}$.
  برای هر
		$b \in \IN$
		به اندازه‌ی کافی بزرگ یک
		$a \in \IN$
		وجود دارد که
		$a/b < \log_{\card{\chi}} \card{\chi'} < (a+1)/b$.
  حال با
		$b$
		بار تکرار کد قبلی یک 
		$(bt, br)$-کد
		برای
		$I'$
		می‌سازیم. از طرفی چون
		$\card{\chi}^1 < \card{\chi'}^b < \card{\chi}^{a + 1}$
		و این‌که هر دو نمونه روی یک مجموعه اطلاعات جانبی تعریف شده‌اند. پس
		$(at, (a+1)r)$-کد
		برای مسئله‌ی
		$I$
		براساس
		$(bt, br)$-کد
		 مسئله‌ی
		 $I'$
		 ساخت. حال با میل دادن
		 $b \rightarrow \infty$
		 و در نتیجه
		 $a \rightarrow \infty$
		 خواهیم داشت
		 $R \in \mathscr{C}_{\chi}$
		  که نتیجه می‌دهد:
		  $\mathscr{A}' \subseteq \mathscr{C}_{\chi}$
		  . چون
		  $\mathscr{C}_{\chi}$
		  بسته است پس
		  $\mathscr{C}_{\chi'} \subseteq \mathscr{C}_{\chi}$
		  با تکرار استدلال بالا به طور مشابه خواهیم داشت:
		  $\mathscr{C}_{\chi} \subseteq \mathscr{C}_{\chi'}$
		  که اثبات را کامل می‌کند \cite{fatemehbook}.
\end{proof}

