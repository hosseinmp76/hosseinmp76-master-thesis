\chapter{کارهای پیشین}
\section{نتایج جدید}

در این فصل نتایج جدید به‌دست‌آمده در پایان‌نامه توضیح داده می‌شود.
در صورت نیاز می‌توان نتایج جدید را در قالب چند فصل ارائه نمود.
همچنین در صورت وجود پیاده‌سازی، بهتر است نتایج پیاده‌سازی را
در فصل مستقلی پس از این فصل قرار داد.

\section{نتایج جدید}\label{sec3}
در این بخش، ابتدا به طور رسمی مسئله کدگذاری منبع انعطاف پذیر را تعریف می‌کنیم. سپس، قضیه اصلی این مقاله را در مورد ارتباط بین مسئله کد انعطاف پذیر خطی (PIC) و کد منبع انعطاف پذیر خطی (PSID) بیان می‌کنیم. توجه داشته باشید که تمام توابع در این مقاله خطی هستند مگر اینکه خلاف آن ذکر شود.
قبل از اینکه شروع به تعریف PSID کنیم، مسئله PIC را هم به صورت دلخواه و هم به صورت خطی بررسی می‌کنیم. به طور خلاصه، مسئله کد انعطاف پذیر می‌گوید که اگر یک نمودار اطلاعات جانبی داده شود (یعنی هر کاربر مقادیر مشخصی از $X_j$ها را می‌داند)، سپس فرستنده اطلاعات بیشتری (یعنی یک عنصر از یک مجموعه متناهی $\Sigma$) ارسال می‌کند به طوری که هر کاربر بتواند برخی از $X_j$ها را از اطلاعاتی که قبلاً داشت (اطلاعات جانبی) و قطعه جدید اطلاعاتی که از فرستنده دریافت کرده است، بازیابی کند. هدف این است که اندازه $\Sigma$ را به حداقل برسانیم تا این کار انجام شود.

به عنوان مثال، وقتی هر کاربر $c_i$ تمام مقادیر $X_j$ها به جز $X_i$ را می‌داند، سپس فرستنده می‌تواند $X_1+X_2+\ldots +X_n$ را ارسال کند. حالا، هر کاربر می‌تواند اطلاعات جانبی خود را از پیام ارسال شده کم کند تا $X_i$ که قبلاً نداشته است، بازیابی کند.

یک روش جایگزین برای نگاه کردن به این راه‌حل این است که اگر $\Sigma =\{z_1, z_2, \ldots, z_k\}$، آنگاه می‌توانیم تمام بردارها در $\mathbb{F}^n$ را به $k$ قسمت تقسیم کنیم؛ به طوری که کسانی که کدگذاری شان $z_i$ است در یک قسمت گروه‌بندی شوند. حال، وقتی فرستنده بردار $X$ را از $\mathbb{F}^n$ انتخاب می‌کند، به تمام مشتریان می‌گوید که $x$ انتخاب شده به کدام قسمت تعلق دارد. سپس وظیفه مشتریان است که به اطلاعات جانبی خود و به گروهی که $X$ به آن تعلق دارد، نگاه کنند و سپس یک ورودی جدید از $X$ را که قبلا نمی‌دانستند یاد بگیرند.

در مسئله کدگذاری منبع انعطاف‌پذیر، هیچ فرستنده‌ای وجود ندارد که به مشتریان بگوید کدام قسمت $X$ انتخاب شده است. به جای آن، مجموعه‌ای از بردارهای $\mathbb{F}^m$ وجود دارد که به آن کتاب کد (همچنین به عنوان یک جدول نامیده می‌شود) گفته می‌شود که $X$ به آن تعلق دارد. سپس، با توجه به اطلاعات جانبی و کتاب کد، هر مشتری باید یک مختصات از $X$ را فراتر از مختصات‌هایی که قبلاً می‌دانست یاد بگیرد. به عنوان مثال، در نمودار اطلاعات جانبی در مثال بالا، فرض کنید که کتاب کد شامل تمام بردارها $(X_1, X_2, \ldots, X_n)$ است به طوری که $X_1 + X_2, \ldots + X_n = 0$. از آنجایی که هر $c_i$ همه $X_j$‌ها به جز $X_i$ را می‌داند، می‌تواند مختصات باقی‌مانده $X_i$ را بازیابی کند.

توجه داشته باشید که در مسئله PIC، هدف کاهش اندازه پیام‌های منتقل شده است در حالی که در مسئله PSIC، هدف افزایش اندازه کتاب کد (جدول قابل استفاده) است. اکنون به طور رسمی یک جدول قابل استفاده را تعریف می‌کنیم.


\begin{definition}[جدول قابل استفاده]\label{def1}
برای یک الفبای داده شده $\mathbb{F}$ و نمودار اطلاعات جانبی $G$، یک جدول قابل استفاده $T$ مجموعه‌ای است با عناصری از $\mathbb{F}^m$ به طوری که برای هر مشتری $c_i$ اگر اطلاعات جانبی آن یکسان است در بعضی از عناصر مختلف، حداقل یک شاخص $i$ وجود دارد که در اطلاعات جانبی او نیست و مقدار یکسانی در تمام آن عناصر دارد. یعنی:
\begin{align*}
    \forall i: \exists j \notin S_i: \forall X, X^\prime: X[S_i] = {X^\prime}[S_i]  \Rightarrow X[j] = {X^\prime}[j]
\end{align*}
ما از اصطلاح جدول استفاده می‌کنیم زیرا عناصر این مجموعه را در یک جدول نمایش می‌دهیم. به این ترتیب، هر ستون یکی از متغیرهای $X_i$ را نشان می‌دهد و هر سطر یک تاپل پیام ممکن است.
\end{definition}

مسئله کدگذاری منبع انعطاف‌پذیر، PSIC، برای یافتن بزرگترین جدول قابل استفاده ممکن برای یک $G$ داده شده است.

\begin{remark}
    ما همیشه می‌توانیم $\mathbb{F}^n$ را به جدول‌های قابل استفاده تقسیم کنیم، به عنوان مثال هر $X \in \mathbb{F}^n$ را به عنوان یک جدول قابل استفاده با یک ردیف در نظر بگیریم.
\end{remark}

\begin{lemma}
    اگر $G$ یک گراف باشد و $\mathbb{F}^m$ به جدول‌های قابل استفاده مجزا تقسیم شود، آنگاه تابع رمزگذاری که بردار $X \in \mathbb{F}^m$ را گرفته و آن را به شاخص جدول قابل استفاده‌ای که $X$ به آن تعلق دارد رمزگذاری می‌کند، یک راه‌حل معتبر برای مسئله PIC برای گراف $G$ است.
\end{lemma}
\begin{proof}
    مشتری $c_i$ را در نظر بگیرید. مشتری $c_i$ $X[S_i]$ را می‌داند. بر اساس تعریف جدول قابل استفاده، شاخص $j$ خارج از $S_i$ وجود دارد که $c_i$ می‌تواند از جدولی که شاخص آن منتقل شده، یاد بگیرد.
\end{proof}

\begin{corollary}\label{cor1}
اگر همه دسته‌بندی‌های پیام ممکن (یعنی $\mathbb{F}^t$) را در $l$ جدول قابل استفاده مختلف تقسیم کنیم، آنگاه می‌توانیم فقط $\lceil \log(l) \rceil$ بیت داده را به کاربران بفرستیم تا شاخص جدول انتخاب شده را پیدا کرده و پیام مورد نظر خود را بیابند.
\end{corollary}

\begin{note}
    در مورد PIC خطی، تابع رمزگذاری $En$ را می‌توان با یک ماتریس توصیف کرد. $\overrightarrow{f}$ توابع خطی هستند. همچنین، اگر سرور $k$ پیام $Y_i = a_{i,1} X_1 + a_{i,2} X_2 \ldots + a_{i,n} X_n$ را ارسال کند، آنگاه
    \begin{equation*}
        En =
        \begin{pmatrix}
            a_{1,1} & a_{1,2} & \cdots & a_{1,n} \\
            a_{2,1} & a_{2,2} & \cdots & a_{2,n} \\
            \vdots  & \vdots  & \ddots & \vdots  \\
            a_{k,1} & a_{k,2} & \cdots & a_{k,n}
        \end{pmatrix}
    \end{equation*}
\end{note}

\begin{definition}[\textbf{جدول قابل استفاده خطی}]
    فرض کنید $\mathbb{F}$ یک میدان متناهی است. یک جدول قابل استفاده $T$ بر روی $\mathbb{F}$ "خطی" نامیده می‌شود اگر ردیف‌های $T$ یک فضای برداری بر روی $\mathbb{F}$ تشکیل دهند.
\end{definition}

مسئله کدگذاری منبع خطی انعطاف‌پذیر، LPSIC، برای یافتن بزرگ‌ترین جدول قابل استفاده خطی ممکن برای یک گراف داده شده $G$ است.


\begin{remark}
    همیشه حداقل یک جدول ممکن خطی وجود دارد، $\{\overrightarrow{0}\}$. در حالت خطی، ما می‌توانیم $\mathbb{F}^n$ را با یک جدول ممکن خطی و هم‌مجموعه‌های آن تقسیم‌بندی کنیم، به سادگی با در نظر گرفتن جدول ممکن خطی و تمامی انتقالات آن. این کار به ما امکان می‌دهد تا تمام $\mathbb{F}^n$ را پوشش دهیم.
\end{remark}


\begin{example}
    در نظر بگیرید
    $G_1$
    و
    $G_2$
    در شکل
    \ref{fig1}.
    برای
    $G_1$
    می‌توانیم دسته‌پیام‌ها را در 4 مجموعه مختلف (جدول‌ها) تقسیم‌بندی کنیم. در این مثال، همه آن‌ها جدول‌های ممکن هستند. برای
    $G_2$
    ما یک جدول ممکن خطی داریم و تنها هم‌مجموعه‌ای که با هم
    $\mathbb{F}_2^3$
    را پوشش می‌دهند:

    \definecolor{myblue}{RGB}{80,80,160}
    \definecolor{mygreen}{RGB}{80,160,80}
    \begin{tikzpicture}[->, >=stealth, auto, semithick]
        % Set the positions of the nodes
        \node[circle, draw=blue, fill=blue!20, inner sep=0pt] (C1) at (0,2) {$C_1$};

        \node[circle, draw=blue, fill=blue!20, inner sep=0pt] (C2) at (1.5,2) {					$C_2$				};
        \node[circle, draw=blue, fill=blue!20, inner sep=0pt] (C3) at (3,2) {$C_3$				};

        \node[circle, draw=green, fill=green!20, inner sep=0pt] (B1) at (0,0) {$B_1$};
        \node[circle, draw=green, fill=green!20, inner sep=0pt] (B2) at (1.5,0) {$B_2$};
        \node[circle, draw=green, fill=green!20, inner sep=0pt] (B3) at (3,0) {$B_3$};


        % Draw the edges
        \draw (C1) -- (B2);
        \draw (C1) -- (B3);

        \draw (C2) -- (B1);
        \draw (C2) -- (B3);



        \draw (C3) -- (B1);
        \draw (C3) -- (B2);

        % Position the parts
        \begin{scope}[on background layer]
            \node[fit=(C1) (C2) (C3), draw=blue, fill=blue!10, rounded corners] {};
            \node[fit=(B1) (B2) (B3), draw=green, fill=green!10, rounded corners] {};
        \end{scope}
    \end{tikzpicture}
    \begin{table}[h]
        \centering
        \begin{tabular}{|c|}
            \hline
            000 \\
            \hline
            001 \\
            \hline
        \end{tabular}
        \caption*{$T_1$}
    \end{table}
    \begin{table}[h]
        \centering
        \begin{tabular}{|c|}
            \hline
            000  \\
            011   \\
            110  \\
            101   \\
            \hline
        \end{tabular}
        \caption*{$T_2$}
    \end{table}
%     \begin{minipage}{0.45\textwidth}
%         \centering

% \end{minipage}

\end{example}


\begin{lemma}
    فرض کنید $G$ یک گراف دوبخشی روی اجزاء $C, B$ باشد. اگر $(En, I, \overrightarrow{f})$ یک کد نمایه‌ای قابل انعطاف خطی برای $G$ باشد، آنگاه یک LPSIC $(W, J, \overrightarrow{g})$ وجود دارد به طوری که $W  = ker(En)$
\end{lemma}

\begin{proof}
    ابتدا، ایده اثبات را مرور می‌کنیم. برای هر مسئله کدگذاری قابل انعطاف خطی، ترکیب‌های خطی که سرور به کاربران پخش می‌کند را در نظر بگیرید. ضرایب جدول ممکن خطی را تشکیل می‌دهند.

    LPSIC $(W, J, \overrightarrow{g})$ به شکل زیر تعریف می‌شود: $W = ker(En), \overrightarrow{g}_i = \overrightarrow{f}_i(\overrightarrow{0}, X[S_i]), J = I$
    $span(W)$ را در نظر بگیرید. اگر تاپل پیام $V$ از این فضای برداری باشد، آنگاه پیام ارسالی همه صفر خواهد بود چون $En \times V = \overrightarrow{0}$. دانستن اینکه $V$ از $span(W)$ است به این معنی است که کدگذاری ارسالی $\overrightarrow{0}$ است و از آنجایی که کاربر $i$-ام می‌تواند $X[I_i]$ را در PIC حدس بزند، می‌تواند $\overrightarrow{f}_i$ را با $\overrightarrow{0}$ به عنوان کد ارسالی به عنوان تابع رمزگشایی خود استفاده کند. پس $span(W)$ یک جدول ممکن خطی تشکیل می‌دهد و بنابراین $span(W)$ یک جدول ممکن خطی است و $(W, J, \overrightarrow{g})$ یک راه‌حل LPSIC است.
\end{proof}

\begin{lemma}
    اگر $(W, J, \overrightarrow{g})$ یک LPSIC برای $G$ باشد، آنگاه یک LPIC $(En, I, \overrightarrow{f})$ وجود دارد به طوری که $span(En) = W^{\bot}$
\end{lemma}

\begin{proof}
    فرض کنید $r = dim(W)$ و $\{V_1, V_2, \ldots, V_{n - r} \}$ یک پایه برای فضای $n - r$ بعدی $W^\bot$ باشد. ماتریس کدگذاری $En$ را به عنوان یک ماتریس $(n - r) \times n $ تعریف کنید که سطرهای آن $V_1, V_2, \ldots, V_{n - r}$ هستند. واضح است که $span(En) = W^\bot$. همچنین، $J = I$ قرار دهید و $\overrightarrow{f}_i(\overrightarrow{R}, En(X[S_i]) = f(En(X[S_i] - R[S_i])) + R[I_i]$ که $\exists! V \in W: V + R = X $.
    اکنون باید اثبات کنیم که کاربر $i$-ام می‌تواند $X[I_i]$ را در هر تاپل پیام بازیابی کند. به بیان دیگر، تاپل‌های پیام متفاوتی با $X[I_i]$‌های متفاوت وجود ندارد که با اطلاعات جانبی یکسان برای کاربر $i$-ام و کد ارسالی یکسان (یعنی $En(X)$) باشد. فرض کنید برعکس است. آنگاه $X_1, X_2, i$ وجود دارد به طوری که $En(X_1) = En(X_2)$ و $X_1[S_i] = X_2[S_i]$ اما $X_1[I_i] \neq X_2[I_i]$. فرض کنید $W_1, \ldots, W_t$ زیرمجموعه‌های $W$ باشند. آنگاه بردارهای یکتایی وجود دارد که $X_1 = V_1 + {V^\prime}_1, X_1 = V_2 + {V^\prime}_2$، که در آن $V_1, V_2 \in W$ و $V^\prime_1 \in W_p, V^\prime_2 \in W_q$.
    بر اساس تعریف جدول ممکن، داریم $En(V_1) = En(V_2) = 0$.
    باید اثبات کنیم که کاربر $i$ همیشه می‌تواند پیام $I_i$ را تشخیص دهد. $En(X_1) = En(X_2)$ به این معنی است که هر دو $X_1, X_2$ از یک زیرمجموعه هستند، یعنی $V^\prime_1 = V^\prime_2$. ما ادعا می‌کنیم که برای هر دو زیرمجموعه متفاوت $W_1, W_2$ داریم: $\forall U_1, U_2: U_1 \in W_1, U_2 \in W_2: En(U_1) \neq En(U_2)$. این بدان معنی است که کاربران می‌توانند زیرمجموعه تاپل پیام را از همه زیرمجموعه‌های دیگر تمیز دهند. فرض کنید $W_j$ زیرمجموعه‌ای باشد که $X_1, X_2$ در آن قرار دارند، آنگاه: $R = U - U^\prime: U, U^\prime \in W_j$. حال، $X_1 - R, X_2 - R$ هر دو در $W$ قرار دارند به این معنی که کاربر می‌تواند بین آنها تمایز قائل شود. اما این یک تناقض است زیرا پس از اضافه کردن $R[I_i]$ به پیام $X[I_i]$، کاربر $i$-ام می‌تواند بین $X_1, X_2$ تمایز قائل شود.
    فقط باقی مانده است تا ادعا را ثابت کنیم. در غیر این صورت فرض کنید. از آنجایی که $En$ خطی است، داریم: $En(U_1) = En(U_2) \Rightarrow En(U_1 - U_2) = 0 \Rightarrow U_1 - U_2 \in W \Rightarrow \exists i: U_1، U_2 \in W_i$. اما ما فرض کردیم $W_1 \neq W_2$.
\end{proof}

\begin{theorem}
    \label{thm1}
    برای هر گراف اطلاعات جانبی $G$، LPSIC $(W, J, \overrightarrow{g})$ دوگان جبری خطی LPIC $(En, I, \overrightarrow{f})$ است به این معنا که:
    \begin{align*}
    (W, J, \overrightarrow{g}) &= \begin{cases}
                                      W = ker(En)\\
                                      I = J \\
                                      \overrightarrow{g}_i(En(X[S_i])) = \overrightarrow{f}_i(\overrightarrow{0}, En(X[S_i])\\
    \end{cases} \\
    (En, I, \overrightarrow{f}) &= \begin{cases}
                                       span(En) = W^{\bot} \\
                                       J = I \\
                                       \overrightarrow{f}_i(\overrightarrow{R}, En(X[S_i]) = g(En(X[S_i] - R[S_i])) + R[I_i]: \exists! V \in W: V + R = X \\
    \end{cases}
    \end{align*}
    و $dim(W) = n - dim(En)$ یعنی اگر $En: \mathbb{F}^n \rightarrow \mathbb{F}^k$ و کد-کتاب $W$ دارای $l$ توپل پیام باشد پس $\log(l) + k = n$
\end{theorem}
\begin{proof}
    این یک پیامد مستقیم از دو لم آخر است. ما فقط باید برابری ابعاد را نشان دهیم. این از این واقعیت ناشی می‌شود که: $dim(W) + dim(W^{\bot}) = n$.
\end{proof}
