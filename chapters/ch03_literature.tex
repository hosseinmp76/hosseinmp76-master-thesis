\chapter{کارهای پیشین}
\label{chapter:literature}

در این فصل به بررسی پژوهش‌های گذشته بر روی مسئله‌ی کدگذاری اندیس منعطف می‌پردازیم.

\section{مقدمه}
همان‌طور که در فصل اول گفته شد، مسئله‌ی کدگذاری اندیس منعطف توسط براهما و فرگولی در سال 2015 در
\cite{pliablefirstpaper}
به عنوان توسعه‌ای از مسئله‌ی کدگذاری اندیس معرفی شد. در این حالت هر گیرنده به جای اینکه به دنبال یک پیام خاص باشد، به دنبال یک پیام جدید است که در مجموعه‌ی اطلاعات جانبی‌ش نباشد.

این مسئله کاربردهایی مختلفی در مسائل حوزه‌های گوناگون دارد. برای مثال در محاسبات توزیع شده
\cite{datashuf}
\transf{
سیستم‌های پیشنهاد دهنده
}{recommendation system}
\cite{8404065}
\transf{
یادگیری توزیع شده
}{
distributed learning
}
\cite{8682270}.
در این فصل به مرور ادبیات پژوهشی 
\picod
می‌پردازیم.

در 
\autoref{sec:3:2}
ضمن دسته بندی مقالات این حوزه مروری اجمالی بر آن‌ها میکنیم.

در 
\autoref{sec:3:3}
به بررسی الگوریتمی تقریبی برای 
\picod
که توسط فرگولی و همکاران در
\cite{song2017polynomialtime}
معرفی شده است می‌پردازیم.
\section{
	مروری اجمالی بر مقالات حوزه
\picod
}
\label{sec:3:2}
سپس در فصل چهار با چهار گونه‌ی مختلف تعریف شده بر اساس کدگذاری اندیس منعطف آشنا خواهیم شد.
در فصل پنج به ارتباط
پس از آن مقالات متعددی روی این مسئله تمرکز کرده اند. پژوهش بر روی مسئله‌ی کدگذاری اندیس منعطف را می‌توان به دسته‌های زیر تقسیم کرد:
\begin{enumerate}
\item 
ساخت یک کد برای دسته‌ای از گراف‌ها و یا ساخت الگوریتم‌های تقریبی/تصادفی برای همه‌ی گراف‌ها
\item 
تعریف مسائل جدید بر پایه‌ی کدگذاری اندیس منعطف(مانند کدگذاری اندیس ترجیحی و ...)
\item 
استفاده از مسئله‌ی کدگذاری اندیس منعطف برای مدل سازی و حل مسائل دیگر
\item 
کران بالا برای حداکثر تعداد ارسال مورد نیاز و کران پایین برای حداقل تعداد ارسال مورد نیاز
\end{enumerate}

روی هر کدام از موضاع بالا مقالات متعددی کار کرده اند. برای مثال
\begin{enumerate}
	\item الگوریتم‌/کد جدید
	\begin{enumerate}
		\item 
		لیو در
		\cite{8278015}
		با استفاده از روش‌های نظریه اطلاعات به اثبات کران‌هایی برای چند خانواده مختلف از گراف‌های اطلاعات جانبی می‌پردازد.
		\item 
	در
	\cite{10313405}
	الگوریتمی برای.
	\item
	در
	\cite{8871209}
	مسئله را فقط برای بخشی از گراف‌های اطلاعات جانبی که
	\lr{complete–S PICOD}
	نام گذاری می‌کنند حل میکنند.
	\item
	در
	\cite{9759449}
	با تعریف یک مسئله‌ی
	\transf{
	بهینه سازی تنک و با رنک پایین}{
	sparse and low-rank optimization
	}
	و حل آن با 
	\transf{
	الگوریتم تصویر کردن تکراری
	}{Alternating Projection Algorithm}
	الگوریتمی کارا برای حل مسئله ارائه میدهند.
	\item
	در
	\cite{8682270}
	با استفاده از
	\picod
	و الگوریتم بر مبنای
	 \transf{
	 تفاوت تحدب
	 }{
	 	difference-of-convex
	 }
	 ارائه می‌دهند که بر اساس نتایج آزمایشگاهی باعث کاهش پهنای باند مورد نیاز در
	 \transf{
	 یادگیری توزیع شده روی دستگاه‌های نهایی
	 }{ON-DEVICE DISTRIBUTED LEARNING}
	 می‌شود.
	 \item
	 در
	 \cite{sasi2019pliable}
	 برای کلاس خاصی از مسئله‌ی
	 \picod
	 که
	 \transf{
	 متوالی
	 }{consecutive}
	 نامیده می‌شود بحث می‌کنند و برای دو حالت اکستریم آن کد اندیس ارائه می‌دهند. سپس در ادامه برای حالت
	 \lr{c-Constrained}
	 نیز کد ارائه می‌دهند.
	 \item 
	 در
	 \cite{8613483}
	 شبیه مقاله قبلی بر روی
	 \lr{Consecutive Complete–S}
	 کار می کنند و با استفاده از اثبات‌های ترکیبیاتی، اثباتی برای وجود کد با طول مناسب ارائه می‌دهند.
	\end{enumerate}
	 
	\item گونه‌های جدید
	\begin{enumerate}
		\item 
		لیو و همکاران در
		\cite{10015670}
		به مسئله‌ی نشتی ناخواسته‌ی اطلاعات در 
		\icod
		و
		\picod
		می‌پردازند. اگر بخشی از پیام‌ها حساس و بقیه غیر حساس باشند یک شنودکننده‌ی متخاصم بر اساس پیام‌های دریافتی چه مقدار داده کسب خواهد کرد؟ این مقدار را با
		\transf{
		نرخ نشت
		}{leakage rate}
		نشان می‌دهیم. در ادامه نرخ نشت بهینه برای مسائل
		\icod
		را اثبات می‌کنند و الگوریتم قطعی برای پیدا کردن آن ارائه می‌دهند و نشان می‌دهند نتیجه‌ی به دست آمده برای
		\icod
		برای
		\picod
		هم برقرار است.
		\item 
		فرگولی در
		\cite{6620405}
		مسئله‌ی
		\picod
		را به دو روش تعمیم می‌دهد. در روش اول هر گیرنده به جای اینکه به دنبال بازیابی یک پیام باشد به دنبال بازیابی
		$t$
		پیام است که به آن
		\lr{MULT-PICOD}
		می‌گوید.
		در روش دوم فرستنده گراف اطلاعات جالبی را در دست ندارد و تنها می‌داند که گیرنده‌ها چند پیام را از پیش به عنوان اطلاعات جانبی دارند(همه‌ی گیرنده‌ها به تعداد برابر پیام دارند.) و آن را
		\lr{OB-PICOD}
		می‌نامد.
		\item 
		لینکی کار فرگولی در مقاله قبلی را در
		\cite{8625330}
		ادامه می‌دهد و روش کدگذاری جدید برای 
		\lr{MULT-PICOD}
		ارائه می‌دهد.
		\item 
		لیو در
		\cite{9173957}
		گونه جدیدی از مسئله را تعریف میکنند که اولا به جای وجود سرور مرکزی تبادل پیام به صورت نامتمرکز انجام می‌شود  و همچنین هر گیرنده تنها یک پیام جدید خارج از اطلاعات جانبی خود بازیابی می‌کند و هیچ دیتایی راجع به بقیه پیام‌ها کسب نمی‌کند. به دلیل سختی این مسئله در ادامه تنها روی یک حالت خاص که اطلاعات جانبی گیرنده‌های به صورت
		\transf{
		جابه‌جایی های چرخشی 
		$s$
		تایی
		}{$s$ circular shifts}
		هست تمرکز می‌کنند.	

	\end{enumerate}
	\item حل مسائل دیگر
		\begin{enumerate}
			\item 
	در
	\cite{Obead_2023}
	با استفاده از  
	\picod
	مسئله‌ی
	\transf{
	بازیابی منعطف و محرمانه دیتا، همراه با اطلاعات جانبی با یک سرور
	}{
	Single-Server Pliable Private Information Retrieval With Side Information
	}
	را حل میکنند.
			\item
	سانگ و فرگولی در
	\cite{8404065}
	به ارتباط 
	\picod
	و ساختن  سیستم‌های پیشنهاد دهنده‌ با توجه به پهنای باند می‌پردازند.
	\item
	در
	\cite{e24081149}
	به ارتباط 
	\icod
	و
	\picod
	با
	\lr{error-correcting codes with multiple interpretations from the
		tree construction of nested cyclic codes}
		می پردازند.
		\item 
		فرگولی در
		\cite{datashuf}
		با استفاده از
		\picod
		به مسئله‌ی
		\transf{
			بازآرایی داده
		}{Data Shuffling}
		که در مسائل محاسبه‌ی توزیع شده ضاهر می‌شود می‌پردازد.
	\item
	\namef{لینکی}{Song, Linqi}
	و فرگولی در
	
	\cite{7176784}
	بررسی‌ای اجمالی بر تاثیر ایده‌ی فکری پشت
	\picod
	بر دسته‌ای از مسائل مخابراتی که
	\transf{
	کدگذاری نوع محتوا
	}{Content-Type Coding}
	نامیده می‌شود می پردازند. لینکی در ادامه در پایان نامه‌ی دکتری خود
	\cite{linqiphd}
	نتایج متعددی در این زمینه می‌گیرد.
		\end{enumerate}
		\item کران بالا و پایین
		\begin{enumerate}
			\item 
			کران بالای
			$\mathcal{O}(\log(n))$
			برای
			\picod
			با کران بالای
			\icod
			که
			$\mathcal{O}(n)$
			است تفاوت فاحشی دارد. لیو و تانینتی در
			\cite{7606849}
		تلاش می‌کنند با پیدا کردن کرانی برای تعداد گیرنده‌هایی که با هر پیام می‌توان ارضا کرد شهودی برای این مسئله بیابند.
			\item 
			در
			\cite{9518120}
			گراف اطلاعات جانبی را با استفاده از 
			\transf{
			هایپرگراف‌ها
			}{hyper graphs}
			مدل‌سازی کرده و کران بالایی برای تعداد ارسال‌ها پیدا میکنند. سپس با استفاده از این کران اثبات می‌کنند که برای بعضی از حالت ها کد با طول خوبی وجود دارد.
			\item
			در ادامه مقاله قبل با روش مشابه‌ای در
			\cite{9965883}
			کران بالا و کران پایین برای
			\picod
			اثبات شده است.
					\item
					\transf{انگ}{ong}
					همکاران در
			\cite{ong2019improved}
			و سپس در
			\cite{8849527}
			تکنیک بسیار متفاوتی در پیش می‌گیرند. به جای بررسی اطلاعات جانبی گیرنده‌ها به 
			\transf{
			گیرنده‌های غایب
		}{Absent Receivers}
		می‌پردازند. اگر هر گیرنده را بر اساس اطلاعات جانبیش شناسایی کنیم، به زیرمجموعه‌های مجموعه‌ی پیام‌ها که به عنوان گیرنده در گراف وجود ندارند گیرنده غایب می‌گوییم. با استفاده از این روش کران پایین جدیدی برای حداقل ارسال مورد نیاز پیدا می‌کنند.
		\end{enumerate}
\end{enumerate}


\section{
	الگوریتمی تقریبی برای مسئله‌ی
\picod
}
\label{sec:3:3}
