\chapter{پیش‌نیازها}
\label{chapter:preliminaries}
\section{مفاهیم اولیه}

دومین فصل پایان‌نامه به طور معمول به معرفی مفاهیمی می‌پردازد که در پایان‌نامه مورد استفاده قرار می‌گیرند.
در این فصل به عنوان یک نمونه، نکات کلی در خصوص نحوه نگارش پایان‌نامه و نیز برخی نکات نگارشی به اختصار توضیح داده می‌شوند.

\section{نحوه‌ی نگارش}

\زیرقسمت{پرونده‌ها}

پرونده‌ی اصلی پایان‌نامه‌ی شما \mycood{thesis.tex}  نام دارد.
به ازای هر فصل از پایان‌نامه، یک پرونده در شاخه‌ی \mycood{chapters} ایجاد نموده
و نام آن را در پرونده‌ی  \mycood{thesis.tex} (در قسمت فصل‌ها) درج نمایید.
پیش از شروع به نگارش پایان‌نامه، بهتر است پرونده‌ی \mycood{front/info.tex} را باز نموده
و مشخصات پایان‌نامه را در آن تغییر دهید.


\زیرقسمت{عبارات ریاضی}

برای درج عبارات ریاضی در داخل متن از \$...\$ و
برای درج عبارات ریاضی در یک خط مجزا از \$\$...\$\$ یا محیط \lr{equation} استفاده کنید. برای مثال $\sum_{k=0}^{n} {n \choose k} = 2^n$ در داخل متن و عبارت زیر
\begin{equation}
\sum_{k=0}^{n} {n \choose k} = 2^n
\end{equation}
در یک خط مجزا درج شده است.
%همان‌طور که در بالا می‌بینید،
%نمایش یک عبارت یکسان در دو حالت درون‌خط و بیرون‌خط می‌تواند متفاوت باشد.
دقت کنید که تمامی عبارات ریاضی، از جمله متغیرهای تک‌حرفی مانند $x$ و $y$ باید در محیط ریاضی
یعنی محصور بین دو علامت \$ باشند.


\زیرقسمت{علائم ریاضی پرکاربرد}

برخی علائم ریاضی پرکاربرد در زیر فهرست شده‌اند.
برای مشاهده‌ی دستور  معادل پرونده‌ی منبع را ببینید.

\begin{itemize}
\item مجموعه‌های اعداد:
$\IN, \IZ, \IZ^+, \IQ, \IR, \IC$
\item مجموعه:
$\set{1, 2, 3}$
\item دنباله‌:
$\seq{1, 2, 3}$
\item سقف و کف:
$\ceil{x}, \floor{x}$
\item اندازه و متمم:
$\card{A}, \setcomp{A}$
\item همنهشتی:
$a \iequiv{n} 1$
یا
$a \equiv 1 \imod{n}$
%\item شمردن (عاد کردن):
%$3 \divs n, 2 \ndivs n$
\item ضرب و تقسیم:
$\times, \cdot, \div$
\item سه‌نقطه‌:
$1, 2, \dots, n$
\item کسر و ترکیب:
${\frac{n}{k}}, {n \choose k}$
\item اجتماع و اشتراک:
$A \cup (B \cap C)$
\item عملگرهای منطقی:
$\neg p \vee (q \wedge r)$

\item پیکان‌ها:
$\rightarrow, \Rightarrow, \leftarrow, \Leftarrow, \leftrightarrow, \Leftrightarrow$
\item عملگرهای مقایسه‌ای:
$\not=, \le, \not\le, \ge, \not\ge$
\item عملگرهای مجموعه‌ای:
$\in, \not\in, \setminus, \subset, \subseteq, \subsetneq, \supset, \supseteq, \supsetneq$

\item جمع و ضرب چندتایی:
$\sum_{i=1}^{n} a_i, \prod_{i=1}^{n} a_i$
\item اجتماع و اشتراک چندتایی:
$\bigcup_{i=1}^{n} A_i, \bigcap_{i=1}^{n} A_i$
\item برخی نمادها:
$\infty, \emptyset, \forall, \exists, \triangle, \angle, \ell, \equiv, \therefore$
\end{itemize}


\زیرقسمت{لیست‌ها}

برای ایجاد یک لیست‌ می‌توانید از محیط‌های «فقرات» و «شمارش» همانند زیر استفاده کنید.

\begin{multicols}{2}
\begin{itemize}
\item مورد اول
\item مورد دوم
\item مورد سوم
\end{itemize}

\begin{شمارش}
\item مورد اول
\item مورد دوم
\item مورد سوم
\end{شمارش}

\end{multicols}


\زیرقسمت{درج شکل}

یکی از روش‌های مناسب برای ایجاد شکل استفاده از نرم‌افزار \lr{LaTeX Draw} و سپس
درج خروجی آن به صورت یک فایل \mycood{tex} درون متن
با استفاده از دستور  \mycood{fig} یا \mycood{centerfig} است. شکل~\ref{شکل:پوشش رأسی} نمونه‌ای از اشکال ایجادشده با این ابزار را نشان می‌دهد.

\begin{شکل}[ht]
\centerfig{cover.tex}{.9}
\شرح{یک گراف و پوشش رأسی آن}
\برچسب{شکل:پوشش رأسی}
\end{شکل}

\bigskip

همچنین می‌توانید با استفاده از نرم‌افزار \lr{Ipe} شکل‌های خود را مستقیما
به صورت \lr{pdf} ایجاد نموده و آن‌ها را با دستورات \mycood{img} یا  \mycood{centerimg}
درون متن درج کنید. برای نمونه، شکل~\ref{شکل:گراف جهت‌دار} را ببینید.


\begin{شکل}[ht]
\centerimg{fox4.png}{6.5cm}
\شرح{نمونه شکل ایجادشده توسط نرم‌افزار \lr{Ipe}}
\برچسب{شکل:گراف جهت‌دار}
\end{شکل}

علاوه بر این می‌توان شکل‌ها را در ویرایشگرهای \lr{Drawio}  ایجاد نمود. در این محیط این امکان فراهم شده است که به صورت مستقیم به یک ویرایشگر \lr{Drawio} متصل شده و شکل‌ها را ویراش کرد. به منظور کافی است بر روی فایل شکل با پسوند \lr{.drawio} دوبار کلیک کرده، شکل را ویرایش کنید و در انتها آن را ذخیره کنید. معادل \lr{pdf} این شکل در همان فولدری که فایل \lr{.drawio} قرار دارد ذخیره می‌شود.
 برای نمونه، شکل~\ref{شکل:شکل drawio} را ببینید.

\begin{شکل}[ht]
\centerimg{fox4.png}{9cm}
\شرح{نمونه شکل ایجاد شده در محیط \lr{Drawio}. برای ویرایش این شکل بر روی
فایل \lr{test.drawio} در فولدر \lr{figs} دوبار کلیک کنید.}
\برچسب{شکل:شکل drawio}
\end{شکل}



\زیرقسمت{درج جدول}

برای درج جدول می‌توانید با استفاده از دستور  «جدول»
جدول را ایجاد کرده و سپس با دستور  «لوح»  آن را درون متن درج کنید.
برای نمونه جدول~\ref{جدول:عملگرهای مقایسه‌ای} را ببینید.

\vspace{1.5em}

\begin{لوح}[ht]
\تنظیم‌ازوسط
\شرح{عملگرهای مقایسه‌ای}

\begin{tabular}{|c|c|}
\خط‌پر
\سیاه عملگر & \سیاه عملیات \\
\خط‌پر \خط‌پر
\mycood{<} & کوچک‌تر \\
\mycood{>} & بزرگ‌تر \\
\mycood{==} &  مساوی \\
\mycood{<>} & نامساوی \\
\خط‌پر
\end{tabular}

\برچسب{جدول:عملگرهای مقایسه‌ای}
\end{لوح}



\زیرقسمت{درج الگوریتم}

برای درج الگوریتم می‌توانید از محیط «الگوریتم» همانند زیر استفاده کنید.

\begin{الگوریتم}{پوشش رأسی حریصانه}
\ورودی گراف $G=(V, E)$
\خروجی یک پوشش رأسی از $G$
%
%\دستور قرار بده $C = \emptyset$  % \توضیحات{مقداردهی اولیه}
%\تاوقتی{$E$ تهی نیست}
%%\اگر{$|E| > 0$}
%%	\دستور{یک کاری انجام بده}
%%\end‌اگر
%\دستور یال دل‌‌خواه
%$uv \in E$
%را انتخاب کن
%\دستور رأس‌های $u$ و $v$ را
%به $C$ اضافه کن
%\دستور تمام یال‌های واقع بر
%$u$
%یا $v$ را از
%$E$ حذف کن
%\While
%\دستور $C$ را برگردان
\end{الگوریتم}


\زیرقسمت{محیط‌های ویژه}

برای درج مثال‌ها، قضیه‌ها، لم‌ها و نتیجه‌ها به ترتیب از محیط‌های
«مثال»، «قضیه»، «لم» و «نتیجه» استفاده کنید.
برای درج اثبات قضیه‌ها و لم‌ها  از محیط «اثبات» استفاده کنید.

تعریف‌های داخل متن را با استفاده از دستور «مهم» به صورت \مهم{تیره‌} نشان دهید.
تعریف‌های پایه‌ای‌تر را درون محیط «تعریف» قرار دهید.

\begin{تعریف}[اصل لانه‌کبوتری]
اگر $n+1$ کبوتر یا بیش‌تر درون  $n$ لانه قرار گیرند، آن‌گاه لانه‌ای
وجود دارد که شامل حداقل دو کبوتر است.
\end{تعریف}




\section{برخی نکات نگارشی}

این فصل حاوی برخی نکات ابتدایی ولی بسیار مهم در نگارش متون فارسی است.
نکات گردآوری‌شده در این فصل به‌ هیچ‌ وجه کامل نیست،
ولی دربردارنده‌ی حداقل مواردی است که رعایت آن‌ها در نگارش پایان‌نامه ضروری به نظر می‌رسد.

\زیرقسمت{فاصله‌گذاری}

\begin{شمارش}

\item
علائم سجاوندی مانند نقطه، ویرگول، دونقطه، نقطه‌ویرگول، علامت سؤال و علامت تعجب % (. ، : ؛ ؟ !)
بدون فاصله از کلمه‌ی پیشین خود نوشته می‌شوند، ولی بعد از آن‌ها باید یک فاصله‌ قرار گیرد. مانند: من، تو، او.
\item
علامت‌های پرانتز، آکولاد، کروشه، نقل قول و نظایر آن‌ها بدون فاصله با عبارات داخل خود نوشته می‌شوند، ولی با عبارات اطراف خود یک فاصله دارند. مانند: (این عبارت) یا \{آن عبارت\}.
\item
دو کلمه‌ی متوالی در یک جمله همواره با یک فاصله از هم جدا می‌شوند، ولی اجزای یک کلمه‌ی مرکب باید با نیم‌فاصله\زیرنویس{«نیم‌فاصله» فاصله‌‌ای مجازی است که در عین جدا کردن اجزای یک کلمه‌ی مرکب از یک‌دیگر، آن‌ها را نزدیک به هم نگه می‌دارد. معمولاً برای تولید این نوع فاصله در صفحه‌کلید‌های استاندارد از ترکیب Shift+Space استفاده می‌شود.}‌‌
 از هم جدا شوند. مانند: کتاب درس، محبت‌آمیز، دوبخشی.
 \item
 اجزای فعل‌های مرکب با فاصله از یک‌دیگر نوشته می‌شوند، مانند: تحریر کردن، به سر آمدن.
\end{شمارش}


\زیرقسمت{شکل حروف}

\begin{شمارش}

\item
در متون فارسی به جای حروف «ك» و «ي» عربی باید از حروف «ک» و «ی» فارسی استفاده شود. همچنین به جای اعداد عربی مانند ٥ و ٦ باید از اعداد فارسی مانند ۵ و ۶ استفاده نمود.
برای این کار، توصیه می‌شود صفحه‌کلید‌ فارسی استاندارد\زیرنویس{\href{http://persian-computing.ir/download/Iranian_Standard_Persian_Keyboard_(ISIRI_9147)_(Version_2.0).zip}{صفحه‌کلید فارسی استاندارد برای ویندوز}، تهیه‌ شده توسط بهنام اسفهبد} را بر روی سیستم خود نصب کنید.
\item
عبارات نقل‌قول‌شده یا مؤکد باید درون علامت نقل قولِ «» قرار گیرند، نه ''``. مانند: «کشور ایران».
\item
کسره‌ی اضافه‌ی بعد از «ه» غیرملفوظ به صورت «ه‌ی» یا «هٔ» نوشته می‌شود. مانند: خانه‌ی علی، دنباله‌ی فیبوناچی.

        تبصره‌: اگر «ه» ملفوظ باشد، نیاز به «‌ی» ندارد. مانند: فرمانده دلیر، پادشه خوبان.

\item
پایه‌های همزه در کلمات، همیشه «ئـ» است، مانند: مسئله و مسئول، مگر در مواردی که همزه ساکن است که در این ‌صورت باید متناسب با اعراب حرف پیش از خود نوشته شود. مانند: رأس، مؤمن.

\end{شمارش}


\زیرقسمت{جدانویسی}

\begin{شمارش}


\item
علامت استمرار، «می»، توسط نیم‌فاصله از جزء‌ بعدی فعل جدا می‌شود. مانند: می‌رود، می‌توانیم.
\item
شناسه‌های «ام»، «ای»، «ایم»، «اید» و «اند» توسط نیم‌فاصله، و شناسه‌ی «است» توسط فاصله از کلمه‌ی پیش از خود جدا می‌شوند. مانند: گفته‌ام، گفته‌ای، گفته است.
\item
علامت جمع «ها» توسط نیم‌فاصله از کلمه‌ی پیش از خود جدا می‌شود. مانند: این‌ها، کتاب‌ها.
\item
«به» همیشه جدا از کلمه‌ی بعد از خود نوشته می‌شود، مانند: به‌ نام و به آن‌ها، مگر در مواردی که «بـ» صفت یا فعل ساخته است. مانند: بسزا، ببینم.
\item
«به» همواره با فاصله از کلمه‌ی بعد از خود نوشته می‌شود، مگر در مواردی که «به» جزئی از یک اسم یا صفت مرکب است. مانند: تناظر یک‌به‌یک، سفر به تاریخ.
%\end{شمارش}
%
%
%\زیرقسمت{جدانویسی مرجح}
%
%\begin{شمارش}

%\item
%اجزای اسم‌ها، صفت‌ها، و قیدهای مرکب توسط نیم‌فاصله از یک‌دیگر جدا می‌شوند. مانند: دانش‌جو، کتاب‌خانه، گفت‌وگو، آن‌گاه، دل‌پذیر.
%
%        تبصره: اجزای منتهی به «هاء ملفوظ» را می‌توان از این قانون مستثنی کرد. مانند: راهنما، رهبر.

\item
علامت صفت برتری، «تر»، و علامت صفت برترین، «ترین»، توسط نیم‌فاصله از کلمه‌ی پیش از خود جدا می‌شوند.
مانند: سنگین‌تر، مهم‌ترین.

        تبصره‌: کلمات «بهتر» و «بهترین» را می‌توان از این قاعده مستثنی نمود.

\item
پیشوندها و پسوندهای جامد، چسبیده به کلمه‌ی پیش یا پس از خود نوشته می‌شوند. مانند: همسر، دانشکده، دانشگاه.

        تبصره‌: در مواردی که خواندن کلمه دچار اشکال می‌شود، می‌توان پسوند یا پیشوند را جدا کرد. مانند: هم‌میهن، هم‌ارزی.

\item
ضمیرهای متصل چسبیده به کلمه‌ی پیش‌ از خود نوشته می‌شوند. مانند: کتابم، نامت، کلامشان.

\end{شمارش}

