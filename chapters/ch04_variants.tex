
\chapter{
    گونه‌های مختلف مسئله
}
\section{
    گونه‌های مختلف مسئله
}
متن کلی


\section{Preferential Pliable Index Coding}


مقاله "Preferential Pliable Index Coding" به معرفی یک نوع جدید از کدگذاری شاخص انعطاف‌پذیر می‌پردازد. این نوع کدگذاری، که با نام کدگذاری شاخص انعطاف‌پذیر ترجیحی (PPICOD) شناخته می‌شود، بر اساس ترجیحات گیرندگان برای پیام‌های ناشناخته طراحی شده است. در این سیستم، هر گیرنده رتبه‌ای ترجیحی به پیام‌های ناشناخته اختصاص می‌دهد.

این مقاله به بررسی تعادل بین طول کد و معیار رضایت‌مندی کلی در میان گیرندگان می‌پردازد. تحقیقات نشان می‌دهد که چگونه می‌توان از طریق تعیین ترجیحات مناسب، به یک توازن بهینه در این دو جنبه دست یافت. محققان روش‌هایی برای محاسبه مرز پارتو، که مجموعه‌ای از زوج‌های طول کد-رضایت‌مندی دست‌یافتنی را نشان می‌دهد، ارائه کرده‌اند.

برای تحقق این موضوع، الگوریتمی حریصانه مبتنی بر پوشش ترجیحی (PrGrCov) پیشنهاد شده است. این الگوریتم به گونه‌ای طراحی شده که در هر تکرار، تعادلی بین تعداد گیرندگان راضی و میانگین معیار رضایت‌مندی برقرار می‌کند. نتایج عددی حاصل از این تحقیق نشان داده است که الگوریتم پیشنهادی توانایی نزدیک شدن به مرز پارتو را دارد و در مقایسه با روش‌های موجود، عملکرد بهتری ارائه می‌دهد.

علاوه بر این، محققان به بررسی تأثیر ترجیحات گیرندگان بر کارایی کدگذاری پرداخته‌اند. آن‌ها نشان داده‌اند که با تنظیم دقیق ترجیحات، می‌توان به بهینه‌سازی کلی در طول کد و رضایت‌مندی دست یافت. این امر نشان‌دهنده قابلیت‌های وسیع این رویکرد جدید در بهبود کارایی سیستم‌های ارتباطی می‌باشد.


\section*{ Private Pliable Index Coding}

در این مقاله، محققان به بررسی یک نوع خاص از مسئله کدگذاری شاخص انعطاف‌پذیر (PICOD) با تمرکز بر حفظ حریم خصوصی پرداخته‌اند. این مطالعه، که در زمینه‌ای نوین و با چالش‌های خاص خود می‌باشد، نشان می‌دهد که چگونه می‌توان ضمن تضمین حفظ اطلاعات خصوصی کاربران، به کدگذاری موثر و بهینه دست یافت.

\textbf{}*{مدل سیستم و چالش‌های آن}
مدل مورد بررسی در این مقاله شامل $n$ کاربر و یک فرستنده مرکزی است، که هر کاربر دارای مجموعه‌ای از اطلاعات جانبی است. تفاوت اصلی این مدل با مدل‌های قبلی در نیاز به حفظ حریم خصوصی است. محققان نشان داده‌اند که در برخی شرایط خاص، ممکن است هیچ کد معتبری برای رعایت محدودیت‌های حریم خصوصی وجود نداشته باشد.

\textbf{}*{نتایج اصلی و روش‌های مورد استفاده}
نتایج اصلی این مقاله شامل ارائه حدود قابل دستیابی و مکالمه‌ای برای مسئله PICOD خصوصی است. این نتایج نشان می‌دهند که طرح‌های قابل دستیابی خطی برای برخی پارامترها بهینه هستند و در برخی موارد تنها به یک انتقال بیشتر نسبت به حد مکالمه نیاز دارند. اثبات‌ها شامل بررسی دقیق ساختار اطلاعات جانبی و تاثیر آن بر امکان کدگذاری است.

\textbf{}*{تاثیرات و کاربردهای تحقیق}
تحقیقات انجام شده در این مقاله، اهمیت زیادی در زمینه‌هایی که حریم خصوصی داده‌ها از اهمیت بالایی برخوردار است دارد. کاربردهای این تحقیق می‌تواند شامل سیستم‌های ارتباطی، شبکه‌های اجتماعی و سیستم‌های توزیع داده باشد که در آن‌ها حفظ حریم خصوصی کاربران از اولویت‌های اصلی است.

\textbf{}*{نتیجه‌گیری و جهت‌های آینده}
این مقاله قدم مهمی در درک و بهبود کدگذاری شاخص انعطاف‌پذیر با تاکید بر حریم خصوصی برداشته است. جهت‌های آینده تحقیق می‌تواند شامل بررسی ساختارهای اطلاعات جانبی پیچیده‌تر، طراحی الگوریتم‌های کدگذاری بهینه‌تر و ارزیابی تاثیر تنظیمات و پارامترهای مختلف بر عملکرد سیستم باشد.



\section*{Decentralized PIC}

این مقاله با عنوان "کدگذاری شاخص انعطاف‌پذیر متمرکز نشده" نوشته تانگ لیو و دانیلا تونینتی، که در آوریل ۲۰۱۹ به آرشیو
arXiv ارسال شده است، یک نوع متفاوت از مسئله کدگذاری شاخص (IC) به نام مسئله کدگذاری شاخص انعطاف‌پذیر متمرکز نشده (PICOD) را معرفی می‌کند.
این مسئله با IC سنتی تفاوت دارد به این صورت که فاقد فرستنده مرکزی است. در عوض، کاربران پیام‌ها را بر اساس اطلاعات جانبی محلی خود به اشتراک می‌گذارند.
این مقاله بیشتر به بررسی ظرفیت دو نوع مسئله PICOD کامل–S متمرکز نشده با m پیام می‌پردازد.

مشارکت‌های کلیدی این مقاله شامل محدودیت‌های معکوس نظری اطلاعات برای دو کلاس از مسائل PICOD متمرکز نشده است: نوع تکمیلی-پیوسته و پیوسته کامل-S PICOD.
نویسندگان نشان می‌دهند که در این موارد، طول کد بهینه برای PICOD متمرکز نشده همانند PICOD متمرکز است، به جز در موارد خاصی که مسئله غیر انعطاف‌پذیر شده و
به یک مسئله IC تقلیل می‌یابد.

نویسندگان از کدهای خطی برداری استفاده می‌کنند، در مقابل کدهای خطی اسکالر مورد استفاده در تنظیمات متمرکز، برای دستیابی به طول کد بهینه در سناریوهای متمرکز نشده.
این مقاله شواهد و تحلیل‌های نظری مفصلی برای این نتایج ارائه می‌دهد و تفاوت‌ها و شباهت‌های بین مسائل PICOD متمرکز و متمرکز نشده را نشان می‌دهد.
یافته‌ها برای درک ظرفیت‌ها و محدودیت‌های سیستم‌های ارتباطی متمرکز نشده، به ویژه در سناریوهایی بدون فرستنده مرکزی، اهمیت دارند.



\section*{Very Pliable Index Coding}

در این مقاله، محققان به بررسی نوع خاصی از مسئله کدگذاری شاخص انعطاف‌پذیر می‌پردازند. این نوع کدگذاری، که بسیار انعطاف‌پذیر نامیده می‌شود، به گیرندگان اجازه می‌دهد هر پیام جدیدی را که از قبل نمی‌دانند رمزگشایی کنند. طراحی کد بهینه در این مدل شامل شناسایی انتخاب هر گیرنده از یک پیام جدید است که نرخ انتقال کلی را به حداقل می‌رساند.

این مقاله فرمول‌بندی‌ای را پیشنهاد می‌دهد که محدودیت‌های رمزگشایی انعطاف‌پذیر را با اجازه دادن به گیرندگان برای رمزگشایی پیام‌های جدید مختلف بسته به محقق شدن پیام‌ها، بیشتر آرام می‌کند. نشان داده شده است که چنین آرام‌سازی هیچ مزیت نرخی را وقتی که از کدهای خطی استفاده می‌شود، ارائه نمی‌دهد اما می‌تواند نرخ‌های به طور قطع بهتری را به طور کلی دستیابی کند.

مقاله به بررسی مدل سیستم برای PICOD قابل توزیع می‌پردازد که شامل کاربران بدون فرستنده مرکزی، پیام‌های مستقل، اطلاعات جانبی برای هر کاربر، و یک پیوند پخش بدون خطا است. هدف یافتن کد معتبر با حداقل طول است. نتایج اصلی شامل دو قضیه برای تعیین طول کد بهینه در موارد مختلف PICODptq قابل توزیع است.

این مقاله نتایج زیر را برقرار می‌کند: نرخ پخش برای کدگذاری شاخص بسیار انعطاف‌پذیر (VP) حداقل یک است. همچنین نشان داده شده است که کدهای خطی می‌توانند برای برخی از موارد کدگذاری شاخص به طور قطع زیر بهینه باشند. در این مقاله، تنظیمات کدگذاری شاخص بسیار انعطاف‌پذیر و کدهای VP بهینه نرخ غیر خطی آن‌ها را ارائه می‌دهیم که به طور قطع از کدهای خطی بهتر عمل می‌کنند.

