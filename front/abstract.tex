
% -------------------------------------------------------
%  Abstract
% -------------------------------------------------------


\begin{وسط‌چین}
\مهم{چکیده}
\end{وسط‌چین}
\بدون‌تورفتگی

یکی از مسائل مهم در مخابرات کاهش حجم ارسالی برای جابه‌جایی مقدار مشخصی اطلاعات است. در واقع گلوگاه سرعت در دنیای صنعت نه توان محاسباتی رایانه‌ها و الگوریتم‌ها بلکه توان جابه‌جایی داده‌ها به صورت فیزیکی است.

یکی از زمینه‌های علمی که به این بهینه‌سازی در انتقال داده، می‌پردازد 
\transf{
	کدگذاری اندیس 
}{
index coding
}
است. اولین مسئله‌ای که در کدگذاری اندیس طرح می‌شود بررسی موقغیتی است که یک ماهواره میخواهد به طور همزمان برای چندین کاربر از یک مجموعه‌ای داده‌ها، داده‌های مختلفی را ارسال کند. اگر نخواهیم تمام داده‌ها را ب صورت پیاپی ارسال کنیم و در نتیجه به اندازه‌ی تعداد درخواست‌ها(و یا اندازه‌ی مجموعه‌ی داده‌ها) زمان صرف کنیم مجبوریم با استفاده از روش‌های مختلف داده‌ها را کدگذاری کنیم.

در زمینه‌ی کدگذاری اندیس و توسیع‌های مختلف آن کارهای متعددی انجام شده است. یکی از توسیع‌ها،
\transf{
کدگذاری اندیس متعطف
}{
pliabe index coding
}
است. در این مسئله گیرنده‌های پیام نسبت به کدگذاری اندیس شرایط کمتر سخت‌گیرانه ای دارند. در این پایان‌‌نامه با استفاده از ابزارهای مختلف ریاضی مانند نظریه گراف و روش‌های احتمالاتی مسائل مختلف این حوزه را بررسی و به ارتباط آن با دیگر شاخه‌ها مانند کدگذار‌ی‌شبکه و رنگ آمیزی گراف‌ها بپردازیم.

\پرش‌بلند
\بدون‌تورفتگی \مهم{کلیدواژه‌ها}:
کدگذاری اندیس قابل انعطاف، کدگذاری اندیس، انتقال داده
\صفحه‌جدید
