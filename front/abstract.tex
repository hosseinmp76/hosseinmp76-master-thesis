
% -------------------------------------------------------
%  Abstract
% -------------------------------------------------------


\begin{وسط‌چین}
\مهم{چکیده}
\end{وسط‌چین}
\بدون‌تورفتگی

یکی از مسائل مهم در مخابرات کاهش حجم ارسالی برای جابه‌جایی مقدار مشخصی اطلاعات است. در واقع گلوگاه سرعت در دنیای صنعت نه توان محاسباتی رایانه‌ها و الگوریتم‌ها بلکه توان جابه‌جایی داده‌ها به صورت فیزیکی است.

یکی از زمینه‌های این بهینه‌سازی انتقال داده کدگذاری اندیس است. مسئله اول کدگذاری اندیس برای زمانی است که یک ماهواره میخواهد به طور همزمان برای چندین کاربر داده‌های مختلفی را ارسال کند. اگر نخواهیم تمام داده‌ها را پیاپی ارسال کنیم و به اندازه‌ی تعداد داده‌ها زمان صرف کنیم مجبوریم با استفاده از روش‌های مختلف داده‌ها را ترکیب و بخش خاصی از آن‌ها را ارسال کنیم.

در این پایان‌‌نامه می‌خواهیم با استفاده از ابزارهای گسسته مانند نظریه گراف و روش‌های احتمالاتی مسائل مختلف این حوزه را بررسی و به ارتباط آن با دیگر شاخه‌ها مانند کدگذار‌ی‌شبکه و رنگ آمیزی گراف‌ها بپردازیم. همچنین سویه‌های مختلف کدگذاری اندیس مانند کدگذاری اندیس قابل انعطاف را نیز بررسی میکنیم.
\lrfootnote{\XeTeX}

\پرش‌بلند
\بدون‌تورفتگی \مهم{کلیدواژه‌ها}:
کدگذاری‌اندیس، انتقال‌داده، کدگذاری‌اندیس‌قابل‌انعطاف
\صفحه‌جدید
