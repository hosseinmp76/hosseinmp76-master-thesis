
% -------------------------------------------------------
%  Abstract
% -------------------------------------------------------


\begin{center}
\مهم{چکیده}
\end{center}
\noindent

یکی از مسائل مهم در مخابرات، کاهش حجم ارسال شده برای جابه‌جایی مقدار مشخصی اطلاعات است. در واقع گلوگاه سرعت در دنیای صنعت، نه توان محاسباتی رایانه‌ها یا زمان اجرای الگوریتم‌ها، بلکه توان جابه‌جایی داده‌ها به صورت فیزیکی است.

یکی از زمینه‌های علمی که به بهینه‌سازی در انتقال داده می‌پردازد\transf{	کدگذاری اندیس}{index coding} است. اولین مسئله‌ای که در کدگذاری اندیس طرح می‌شود، بررسی موقعیتی است که یک ماهواره می‌خواهد از مجموعه‌ای از داده‌ها، برای چندین کاربر به طور همزمان، داده‌های مختلفی را ارسال کند. اگر نخواهیم تمام داده‌ها را به صورت پیاپی ارسال کنیم و در نتیجه به اندازه‌ی تعداد درخواست‌ها(یا اندازه‌ی مجموعه‌ی داده‌ها) زمان صرف کنیم مجبوریم با استفاده از روش‌های مختلف داده‌ها را کدگذاری کنیم.

در زمینه‌ی کدگذاری اندیس و نسخه‌های مختلف آن، کارهای متعددی انجام شده است. یکی از این نسخه‌ها، \transf{کدگذاری اندیس منعطف}{pliabe index coding} است. در این مسئله گیرنده‌ها، نسبت به کدگذاری اندیس شرایط کمتر سخت‌گیرانه‌ای دارند. در این پایان‌‌نامه، با استفاده از ابزارهای مختلف ریاضی مانند نظریه گراف و روش‌های احتمالاتی، مسائل مختلف این حوزه را بررسی کرده و به ارتباط آن با دیگر شاخه‌ها مانند کدگذار‌ی‌شبکه و ذخیره‌سازی توزیع‌شده می‌پردازیم و دوگان کدگذاری اندیس منعطف را معرفی و دوگانی آن‌ها را اثبات می‌کنیم.

\bigskip
\noindent \مهم{کلیدواژه‌ها}:
کدگذاری اندیس قابل انعطاف، کدگذاری اندیس، انتقال داده
\newpage
