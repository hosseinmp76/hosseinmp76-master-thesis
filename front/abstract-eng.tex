
% -------------------------------------------------------
%  English Abstract
% -------------------------------------------------------


\pagestyle{empty}

\begin{latin}

\begin{center}
\textbf{Abstract}
\end{center}
\baselineskip=.8\baselineskip
One of the important issues in telecommunications is reducing the transmission volume for transferring a specific amount of information. In fact, the bottleneck in speed within the industrial world is not the computational power of computers and algorithms, but rather the capability of physically transferring data.

One of the scientific fields that deals with optimization in data transmission is index coding\lrfootnote{ICOD}. The first issue presented in index coding is examining the scenario where a satellite aims to simultaneously send different data to multiple users from a set of data. If we do not want to send all the data sequentially and consequently spend as much time as the number of requests (or the size of the data set), we are compelled to encode the data using various methods.

In the field of index coding and its various extensions, numerous works have been conducted. One of these extensions is pliable index coding\lrfootnote{PICOD}. In this problem, the receivers have less strick conditions regarding index coding. In this thesis, we examine various issues in this domain using different mathematical tools such as graph theory and probabilistic methods and explore its connections with other parts of matematics like network coding and distributed storage and we prove the duality of pliable index coding and pliable source coding.

\bigskip\noindent\textbf{Keywords}:
pliable index coding, index coding, data-transfer 

\end{latin}
